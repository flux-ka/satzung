% !TeX spellcheck = de_DE
\documentclass[11pt]{article}
\usepackage[ngerman]{babel}
\usepackage [T1]{fontenc}
\usepackage{biolinum}
\usepackage[a4paper]{geometry}
\usepackage{csquotes}

\usepackage[printwatermark]{xwatermark}
\usepackage{xcolor}
\usepackage{graphicx}

%\newwatermark[pages=1-9,color=black!10,angle=45,scale=3,xpos=0,ypos=0]{Vorläufige Fassung}

\biolinum

\renewcommand{\thesection}{§ \arabic{section}}

\begin{document}
\begin{center}
	\Large
	\textbf{
		Satzung der Hochschulgruppe ``Flux''}\\
	\normalsize
	(nur aus steuerlichen Gründen notwendige Bestimmungen)\\
	\vspace{10mm}
	Fassung vom 05.07.2024
	\vspace{10mm}
\end{center}

\section{Name, Sitz, Geschäftsjahr}
\begin{enumerate}
	\item Name der Hochschulgruppe ist ``Flux'', abgekürzt ``flux''.
	\item Die Hochschulgruppe hat ihren Sitz in Karlsruhe.
	\item Das Geschäftsjahr ist das Kalenderjahr.
\end{enumerate}
\section{Zwecke und Ziele}
\label{zweck}
Zweck der Hochschulgruppe ist:
\begin{itemize}
	\item die Förderung der Studentenhilfe. [Besonders bezogen auf Bildung und Förderung im Bereich von neurodivergenten Studierenden]
	\item die Förderung der Hilfe für Behinderte.
\end{itemize}
Der Satzungszweck wird verwirklicht insbesondere durch:
\begin{itemize}
	\item Zusammenarbeit mit anderen gemeinnützigen Körperschaften, Organisationen, oder Personen, die einen zum oben genannten Zweck kompatibles Ziel verfolgen.
	\item Organisation von Vorträgen oder anderen Lehrveranstaltungen zum oben genannten Zweck, oder Aufruf zu vergleichbaren Veranstaltungen.
	\item Einsatz für den oben genannten Zweck an verschiedenen offiziellen Stellen des KIT und universitätspolitischen Organen.
	\item Unterstützung Studierender bezüglich verschiedener Belange gegenüber des KIT wie z.B. die Beantragung eines Nachteilsausgleiches oder das Stellen von Härtefallanträgen
\end{itemize}
Insbsondere möchte diese Hochschulgruppe einen sicheren Raum für neurodivergente Menschen bieten und diese unterstützen. Dies kann beispielsweise durch die Bereitstellung eines Raums für Austausch oder die gegenseitige Unterstützung bezüglich universitärer Belange erfolgen.

\section{Gemeinnützigkeit}
Die Hochschulgruppe verfolgt ausschließlich und unmittelbar gemeinnützige Zwecke im Sinne
des Abschnitts ''Steuerbegünstigte Zwecke'' der Abgabenordnung. Sie ist selbstlos tätig und verfolgt
nicht in erster Linie eigenwirtschaftliche oder gewerbliche Zwecke.\\\\
Mittel der Hochschulgruppe dürfen nur für die satzungsmäßigen Zwecke verwendet werden. Die
Mitglieder oder andere Personen erhalten keine Zuwendungen aus Mitteln der Körperschaft.\\\\
Es darf keine Person durch Ausgaben, die dem Zweck der Körperschaft fremd sind, oder durch
unverhältnismäßig hohe Vergütungen begünstigt werden.
\section{Mitgliedschaft: Arten, Beginn, Ende}
\begin{enumerate}
	\item Mitglieder der Hochschulgruppe dürfen nur natürliche Personen werden.
	\item Mitglied kann jede natürliche Person werden, die sich mit dem Zweck sowie den in \ref{zweck} genannten Zielen identifiziert und bereit ist, sich aktiv an der Verwirklichung dieser zu beteiligen.
	\item Ein Anspruch auf Aufnahme in die Hochschulgruppe besteht nicht. Die Mitgliedschaft ist
	      schriftlich beim Vorstand zu beantragen. Entscheidung über Aufnahme erfolgt durch die Mitgliederversammlung, wobei der Vorstand ein Vetorecht bezüglich jeder Aufnahme inne hat. Die Mitgliedschaft beginnt sobald das Mitglied über die Aufnahme benachrichtigt wurde. Das Mitglied wird hierüber schriftlich oder mündlich informiert.
	\item Über die ablehnende Entscheidung wird die antragstellende Person schriftlich unterrichtet.
	\item Die Mitgliedschaft in der Hochschulgruppe ist unteilbar. Es können nicht mehrere Personen
	      gemeinsam eine Mitgliedschaft erwerben.
	\item Die Mitgliedschaft endet durch Tod, Ausschluss, oder schriftliche Kündigung, die an den Vorstand zu richten ist.
	      \begin{enumerate}
		      \item Ausschluss eines Mitgliedes erfolgt durch die Mitgliederversammlung und kann nur durch eine zwei Drittel Mehrheit erfolgen. Der Ausschluss erfolgt unmittelbar.
		      \item Mit Ende der Mitgliedschaft erfolgt automatisch der Rücktritt von sämtlichen Ämtern, die das Mitglied inne hat.
		      \item Erscheint eine Person länger als 2 Jahre nicht zu Mitgliederversammlungen hat der Vorstand das Recht, diese Person ohne Entscheidung der Mitgliederversammlung aus der Hochschulgruppe auszuschließen.
		      \item Gegen den Ausschluss kann beim Vorstand binnen 14 Tagen Widerspruch eingelegt werden.
	      \end{enumerate}
\end{enumerate}
\section{Organe}
Organe der Hochschulgruppe sind:
\begin{enumerate}
	\item Die Mitgliederversammlung
	      \begin{enumerate}
		      \item Die Mitgliederversammlung setzt sich aus mindestens fünf bis maximal allen Mitgliedern zusammen.
		      \item Die Mitgliederversammlung kann jederzeit vom Vorstand einberufen werden. Dazu sind alle Mitglieder mindestens 72 Stunden vor der Versammlung schriftlich zu informieren. Sollten folgende Punkte zur Entscheidung stehen, müssen diese mit Einberufung der Mitgliederversammlung bekannt gegeben werden:
		            \begin{itemize}
			            \item Der Ausschluss eines Mitgliedes
			            \item Die Änderung der Satzung
			            \item Die Wahl des Vorstandes
			            \item Die Auflösung der Hochschulgruppe
		            \end{itemize}
		      \item Jedes Mitglied kann den Vorstand jederzeit schriftlich dazu auffordern die Mitgliederversammlung einzuberufen. Falls der Vorstand nicht binnen einer Woche ablehnt, wird die Mitgliederversammlung automatisch zum vom Mitglied genannten Zeitpunkt einberufen. Die Benachrichtigung aller Mitglieder muss mindestens 72 Stunden vorher durch das auffordernde Mitglied erfolgen. Eine Ablehnung durch den Vorstand ist von diesem schriftlich zu begründen.
		      \item Jedes Mitglied ist berechtigt bis 24 Stunden vor der Mitgliederversammlung Tagesordnungspunkte auf der Tagesordnung zu ergänzen.
		      \item Die Mitgliederversammlung ist beschlussfähig genau dann wenn sie ordnungsgemäß einberufen wurde.
		      \item Jede beschlussfähige Zusammenkunft der Mitgliederversammlung ist zu protokollieren, außer es wurde von der Mitgliederversammlung einstimmig eine explizite Ausnahme für genau diese Zusammenkunft getroffen. Es können keine generellen Ausnahmen von dieser Regel beschlossen werden.
		            \begin{itemize}
			            \item Aus dem Protokoll müssen mindestens die getroffenen Entscheidungen sowie die Stimmenverteilung hervorgehen.
			            \item Das Protokoll ist allen Mitgliedern binnen 14 Tagen schriftlich zur Kenntnis zu geben.
		            \end{itemize}
		      \item Die Mitgliederversammlung kann Satzungsänderungen beschließen. Dazu ist eine zwei Drittel Mehrheit notwendig.
		      \item Bei jeder Mitgliederversammlung kann der/die Vorsitzende, der/die Finanzreferent:in sowie deren Stellvertretung neu gewählt werden. Dazu ist eine zwei Drittel Mehrheit notwendig.
	      \end{enumerate}
	\item Der Vorsitz
	      \begin{enumerate}
		      \item Der Vorsitz ist die primäre Ansprechperson der Hochschulgruppe.
		      \item Der Vorsitz kann bei jeder Mitgliederversammlung mit einer zwei Drittel Mehrheit gewählt werden. Eine Wahl erfolgt bis zum Ende der Mitgliedschaft oder bis zur nächsten Wahl.
		      \item Der Vorsitz kann nicht die selbe Person wie der/die Finanzreferent:in sein.
		      \item Der Vorsitz kann jederzeit unbegründet zurücktreten. Nach Rücktritt ist möglichst zeitnah eine Mitgliederversammlung einzuberufen, um den Posten neu zu besetzen. Solange fällt die Position an den stellvertretenden Vorsitz.
		      \item Der Vorsitz ist dafür verantwortlich die Kasse mindestens ein mal pro Semester zu prüfen und spricht anschließend gegenüber der Mitgliederversammlung eine Entlastung des/der Finanzreferent:in aus.
	      \end{enumerate}
	\item Der stellvertretende Vorsitz\\
	      Der stellvertretende Vorsitz übernimmt den Posten des Vorsitzes, falls der Vorsitz nicht mehr Mitglied der Hochschulgruppe ist, zurücktritt oder andere triftige Gründe vorliegen. Der stellvertretende Vorsitz kann bei jeder Mitgliederversammlung mit einer zwei Drittel Mehrheit gewählt werden. Eine Wahl erfolgt bis zum Ende der Mitgliedschaft oder bis zur Neuwahl.
	\item Der/Die Finanzreferent:in
	      \begin{enumerate}
		      \item Der/Die Finanzreferent:in ist für sämtliche finanziellen Fragen und Belange der Hochschulgruppe außer der Kassenentlastung zuständig.
		      \item Der/Die Finanzreferent:in kann bei jeder Mitgliederversammlung mit einer zwei Drittel Mehrheit gewählt werden. Eine Wahl erfolgt bis zum Ende der Mitgliedschaft oder bis zur Neuwahl.
		      \item Der/Die Finanzreferent:in kann nicht die selbe Person wie der Vorsitz sein.
		      \item Der/Die Finanzreferent:in kann unbegründet zurücktreten nachdem eine Kassenentlastung durch den Vorsitz erfolgt ist. Nach Rücktritt ist möglichst zeitnah eine Mitgliederversammlung einzuberufen,  um den Posten neu zu besetzen. Solange fällt die Position an den/die stellvertretenden Finanzreferent:in.
	      \end{enumerate}
	\item Der/Die stellvertretende Finanzreferent:in\\
	      Der/Die stellvertretende Finanzreferent:in übernimmt den Posten des/der Finanzreferent:in, falls der/die Finanzreferent:in nicht mehr Mitglied der Hochschulgruppe ist, zurücktritt oder andere triftige Gründe vorliegen. Der/Die stellvertretende Finanzreferent:in kann bei jeder Mitgliederversammlung mit einer zwei D
	      rittel Mehrheit gewählt werden. Eine Wahl erfolgt bis zum Ende der Mitgliedschaft oder bis zur Neuwahl.
	\item Der Vorstand
	      \begin{enumerate}
		      \item Der Vorstand besteht immer genau aus dem Vorsitz und Finanzreferent:in. Sobald diese Personen sich ändern, ändert sich automatisch auch der Vorstand.
		      \item Beide Mitglieder des Vorstandes sind berechtigt im Namen des Vorstandes zu sprechen und zu handeln.
	      \end{enumerate}
\end{enumerate}
\section{Beiträge}
Es werden keine Mitgliedsbeiträge oder anderweitige Gebühren erhoben. Die Mitgliedschaft und Arbeit in und für die Hochschulgruppe erfolgt ehrenamtlich und wird nicht vergütet.
\section{Rechte und Pflichten der Mitglieder}
Jedes Mitglied hat das Recht zum/zur Vorsitzende:n, Finanzreferent:in oder deren Stellvertretung gewählt zu werden. Jedes Mitglied hat das Recht an allen Veranstaltungen der Hochschulgruppe, insbesondere der Mitgliederversammlung, teilzunehmen. Ein Ausschluss von bestimmten Veranstaltungen kann nur durch einen Ausschluss aus der gesamten Hochschulgruppe erfolgen.\\
Mitglieder haben keine bestimmten Pflichten.
\section{Satzungsänderungen}
Satzungsänderungen können ausschließlich durch die Mitgliederversammlung erfolgen. Dazu ist eine geänderte Fassung der Satzung schriftlich in digitaler Form vorzulegen. Zur Änderung der Satzung ist eine zwei Drittel Mehrheit erforderlich.
\section{Auflösung oder Aufhebung der Hochschulgruppe}
Die Auflösung der Hochschulgruppe erfolgt:
\begin{itemize}
	\item Jederzeit durch einstimmigen Beschluss der Mitgliederversammlung.
	\item Automatisch, falls zwei Monate nach Rücktritt des Vorsitzes die Position nicht durch eine Neuwahl neu besetzt wurde.
\end{itemize}
%TODO: ändern wenn wir n Verein haben!!
Bei Auflösung des Vereins oder Wegfall steuerbegünstigter Zwecke fällt das Vermögen an den ``Entropia e.V.'', der es ausschließlich für gemeinnützige,
mildtätige oder kirchliche Zwecke zu verwenden hat.
\section{Datenschutz}
\begin{enumerate}
	\item Unter Beachtung der gesetzlichen Vorgaben und Bestimmungen der EU-Datenschutz-
	      Grundverordnung (DSGVO) und des Bundesdatenschutzgesetzes (BDSG) werden zur Erfüllung
	      der Zwecke und Aufgaben der Hochschulgruppe personenbezogene Daten über persönliche und
	      sachliche Verhältnisse der Mitglieder der Hochschulgruppe erhoben und in dem
	      hochschulgruppeneigenen EDV-System gespeichert, genutzt und verarbeitet.
	\item Mit dem Betritt eines Mitgliedes nimmt die Hochschulgruppe alle für die Mitgliedschaft in der
	      Hochschulgruppe relevanten Daten (Name, Studierendenstatus, E-Mail Adresse) auf. Diese Informationen
	      werden in dem hochschulgruppeneigenen EDV-System gespeichert. Die personenbezogenen
	      Daten werden dabei durch geeignete technische und organisatorische Maßnahmen vor der
	      Kenntnisnahme Dritter geschützt.
	\item Sonstige Informationen zu den Mitgliedern und Informationen über Nichtmitglieder werden
	      grundsätzlich nur verarbeitet oder genutzt, wenn sie zur Förderung des Zwecks nützlich sind
	      (wie etwa Telefon- und E-Mail-Adresse) und keine Anhaltspunkte bestehen, dass die betroffene
	      Person ein schutzwürdiges Interesse hat, das der Verarbeitung oder Nutzung entgegensteht.
	      Absatz 2 Satz 3 gilt entsprechend.
	\item Jedes Mitglied hat das Recht darauf,
	      \begin{enumerate}
		      \item Auskunft über die zu seiner Person gespeicherten Daten zu erhalten,
		      \item dass die zu seiner Person gespeicherten Daten berichtigt werden, wenn sie unrichtig sind,
		      \item dass die zu seiner Person gespeicherten Daten gesperrt werden, wenn sich bei behaupteten
		            Fehlern weder deren Richtigkeit noch deren Unrichtigkeit feststellen lässt,
		      \item dass die zu seiner Person gespeicherten Daten gelöscht werden, wenn die Speicherung
		            unzulässig war oder die Zwecke für die sie erhoben und gespeichert wurden nicht mehr
		            notwendig sind,
		      \item der Verarbeitung seiner personenbezogenen Daten zu widersprechen,
		      \item seine Daten in einem strukturierten, gängigen und maschinenlesbaren Format zu erhalten.
	      \end{enumerate}
	\item Den Organen der Hochschulgruppe, allen Mitarbeitern oder sonst für die Hochschulgruppe
	      Tätigen ist es untersagt, personenbezogene Daten unbefugt zu anderen als den zur jeweiligen
	      Aufgabenerfüllung gehörenden Zweck zu verarbeiten, bekannt zu geben, Dritten zugänglich zu
	      machen oder sonst zu nutzen. Diese Pflicht besteht auch über das Ausscheiden der oben
	      genannten Personen aus der Hochschulgruppe hinaus.
\end{enumerate}
\section{Gültigkeit dieser Satzung}
Diese Satzung ist ab sofort gültig.

\end{document}
